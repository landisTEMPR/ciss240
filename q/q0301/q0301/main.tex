%-*-latex-*-
%-*-latex-*-
\newcommand\COURSE{ciss240}
\newcommand\ASSESSMENT{q0603}
\newcommand\ASSESSMENTTYPE{Quiz}
\newcommand\POINTS{\textwhite{xxx/xxx}}

\input{myquizpreamble}
\input{yliow}
\input{\COURSE}
%-*-latex-*-

%-*-latex-*-

%-*-latex-*-
\renewcommand\TITLE{\ASSESSMENTTYPE \ \ASSESSMENT}

\renewcommand\EMAIL{}
\renewcommand\AUTHOR{bglandis1@cougars.ccis.edu}

\textwidth=6in
\begin{document}
\topmatterthree

This is a closed-book, no compiler, 5 minute quiz.

%------------------------------------------------------------------------------
\nextq
Write one \cpp\ statement that
declares an integer variable with the name
\verb!i! and initialize it with the value of \verb!0!.
\\
\ANSWER
\begin{answercode}
int i = 0;
\end{answercode}
(What must a \cpp\ statement end with?)

%------------------------------------------------------------------------------
\nextq
Write one \cpp\ statement that declares an integer variable \verb!j!
without initialization, and then write another statement that
assigns \verb!0! to \verb!j!.
\\
\ANSWER
\begin{answercode}
int j;
j = 0;
\end{answercode}

%------------------------------------------------------------------------------
\nextq
Write \textit{one} \cpp\ statement that declares two integer variables,
one called \texttt{x} and another called \texttt{y}.
Initialize \texttt{x} with \texttt{0} and \texttt{y} with \texttt{1}.
\\
\ANSWER
\begin{answercode}
int x;
int y;
x = 0;
y = 1;
\end{answercode}

%------------------------------------------------------------------------------
\nextq
Before the following statement is executed,
integer variable \verb!x! has a value of 5
and integer variable \verb!y! has a value of 2.
\begin{console}[fontsize=\footnotesize]
x = x + y;
\end{console}
What are the values of \verb!x! and \verb!y! after the above statement is
executed?
Enter exactly one space between the two values.
In other words, if you think \verb!x! is 111 and \verb!y! is 222,
enter \verb!111 222!.
\\
\ANSWER
\begin{answercode}
7 2
\end{answercode}

%------------------------------------------------------------------------------
\nextq
Before the following statements are executed,
the integer variable \verb!x! has a value of 5,
the integer variable \verb!y! has a value of 2, and
the integer variable \verb|z| has a value of 4.
\begin{console}[fontsize=\footnotesize]
y = y + 1;
x = x + y + z;
z = x + z / y;
\end{console}
What are the values of \verb!x!, \verb!y! and \verb!z!
after the above statements are executed?
Enter exactly one space between two values.
In other words, if you think \verb!x! is 111, \verb!y! is 222 and \verb!z! is 333,
you enter \verb!111 222 333!.
\\
\ANSWER
\begin{answercode}
9 3 10
\end{answercode}

%------------------------------------------------------------------------------
\nextq
The integer variable \verb!n! is already declared.
Write one \cpp\ statement to get an integer input from the
user (via the keyboard) and give this integer value to \verb!n!.
\\
\ANSWER
\begin{answercode}
std::cin >> n;
\end{answercode}

%------------------------------------------------------------------------------
\nextq
Write the following statements.
Declare an integer variable \verb!i! (without initialization).
Get an integer input from the user (via the keyboard) and store that
value in \verb!i!.
Declare an integer variable \verb!d0!
and initialize it with the rightmost digit of \verb!i!.
(Example: The \lq\lq rightmost digit" of \verb!135246! is \verb!6!.)
Print the value stored in \verb!d0! followed by the newline character.
\\
\ANSWER
\begin{answercode}
int i;
std::cin >> i;
int d0 = i % 10;
std::cout << d0 << std::endl;
\end{answercode}

%------------------------------------------------------------------------------
%-*-latex-*-
\newpage
\input{instructions}
\end{document}

