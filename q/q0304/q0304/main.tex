%-*-latex-*-
%-*-latex-*-
\newcommand\COURSE{ciss240}
\newcommand\ASSESSMENT{q0603}
\newcommand\ASSESSMENTTYPE{Quiz}
\newcommand\POINTS{\textwhite{xxx/xxx}}

\input{myquizpreamble}
\input{yliow}
\input{\COURSE}
%-*-latex-*-

%-*-latex-*-

%-*-latex-*-
\renewcommand\TITLE{\ASSESSMENTTYPE \ \ASSESSMENT}

\renewcommand\EMAIL{}
\renewcommand\AUTHOR{bglandis1@cougars.ccis.edu}

\textwidth=6in
\begin{document}
\topmatterthree

This is a closed-book, no compiler, 5 minute quiz.

%------------------------------------------------------------------------------
\nextq
In the following
the repeating chunk of code repeats 4 times (i.e. it appears 5 times).
{\small
\begin{console}[commandchars=\@\{\}]
int i = 10, s = 0;

s = s + i;
i = i - 2;

s = s + i;
i = i - 2;

...
\end{console}
What is the final value of \verb!i! and the final value of \verb!s!?
If you think the final value of \verb!i! is \verb!111!
and the final value of \verb!s! is \verb!222!, write \verb!111 222!
with exactly one space between the two values.
\\
\ANSWER
\begin{answercode}
0 30
\end{answercode}

%------------------------------------------------------------------------------
\nextq
Which of the following are valid variable names?
If (a),(b),(e),(f) are the only valid variable names, write \verb!a b e f!
in that order and with exactly one space between two consecutive letters.
\begin{enumerate}[nosep,label=(\alph*)]
\item  \hspace{0.5cm}\texttt{wassup}
\item  \hspace{0.5cm}\texttt{noway!}
\item  \hspace{0.5cm}\texttt{as\_far}
\item  \hspace{0.5cm}\texttt{as3 far4}
\item  \hspace{0.5cm}\texttt{5as\_6far}
\item  \hspace{0.5cm}\texttt{\_as\_far\_}
\item  \hspace{0.5cm}\texttt{gimme\$}
\item  \hspace{0.5cm}\texttt{RETURN}
\end{enumerate}
\ANSWER
\begin{answercode}
a b e f
\end{answercode}

%------------------------------------------------------------------------------
\nextq
\tf. For the following write T for true and F for false (ignore M).
\\
The output of the following code fragment
\begin{console}[fontsize=\footnotesize]
int inheritance_from_uncle = 1000;
int amt_of_$_in_wallet = 10;
int total_wealth = inheritance_from_uncle + amt_of_$_in_wallet;
std::cout << total_wealth << '\n';
\end{console}
is
\begin{console}[fontsize=\footnotesize]
1010
\end{console}
\ANSWER
\begin{answercode}
F
\end{answercode}

%------------------------------------------------------------------------------
\nextq
Here's a code fragment
\begin{console}[fontsize=\footnotesize]
int i = 0, j = 0;

j = j + i;
i = i + 1;

j = j + i;
i = i + 1;

j = j + i;
i = i + 1;
\end{console}
Note that there is a chunk of repeating code that appears 3 times.
If the goal is to compute
$0 + 1 + 2 + 3 + 4 + 5 + 6 + 7 + 8 + 9$
and store this value in variable \verb!j!,
how many times must the repeating chunk of code appear?
If you think $0 + 1 + 2 + 3 + 4 + 5 + 6 + 7 + 8 + 9$
cannot be computed by repeat
the above chunk of code, write ERROR.
\\
\ANSWER
\begin{answercode}
10
\end{answercode}

%------------------------------------------------------------------------------
%-*-latex-*-
\newpage
\input{instructions}
\end{document}

