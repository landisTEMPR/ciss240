%-*-latex-*-
%-*-latex-*-
\newcommand\COURSE{ciss240}
\newcommand\ASSESSMENT{q0603}
\newcommand\ASSESSMENTTYPE{Quiz}
\newcommand\POINTS{\textwhite{xxx/xxx}}

\input{myquizpreamble}
\input{yliow}
\input{\COURSE}
%-*-latex-*-

%-*-latex-*-

%-*-latex-*-
\renewcommand\TITLE{\ASSESSMENTTYPE \ \ASSESSMENT}

\renewcommand\EMAIL{}
\renewcommand\AUTHOR{bglandis1@cougars.ccis.edu}

\textwidth=6in
\begin{document}
\topmatterthree

This is a closed-book, no compiler, 5 minute quiz.

First trace the following program and write down the output.
The console window output of the following program
\begin{console}[fontsize=\small]
#include <iostream>

int main()
{
    std::cout << 'I' << 't' << "was a" << '\n' << "dark\n"
              << std::endl << std::endl << "and  \"stormy\"" << ' ' << " "
              << "night\n";

    return 0;
}
\end{console}
is (use one square for each printed character):

\begin{python}
import string
#def ph(c):
#    s = string.ascii_letters + string.digits + string.punctuation
#    return r'{\vphantom{%s}\texttt{%s}}' % (s, c)

from latextool_basic import *
m = [['I','t','w','a','s','','a'] + ['' for i in range(21-7)],
     ['d','a','r','k'] + ['' for i in range(21-4)],
     ['' for i in range(21)],
     ['' for i in range(21)],
     list('and  "stormy"  night '),
     ['' for _ in range(21)],
]

# replace with spaces
m = [[' ' for c in row] for row in m]
m[0][0] = 'I'

p = Plot()
C = table2(p, m, rowlabel='x', collabel='y',
           do_not_plot=True,rownames=[], colnames=[])

import string
def f(r, c, ch, background='blue!20', vphantom=string.printable):
    x0,y0 = C[r][c].bottomleft()
    x1,y1 = C[r][c].topright()
    return Rect(x0=x0, y0=y0, x1=x1, y1=y1,
                background=background,
                label=ph(ch),
                linewidth=0)

x0,y0 = C[0][2].bottomleft()
x1,y1 = C[0][2].topright()
p += Rect(x0=x0, y0=y0, x1=x1, y1=y1,
          background='blue!20',label=r'\texttt{ }', linewidth=0)

#x0,y0 = C[0][6].bottomleft()
#x1,y1 = C[0][6].topright()
#p += Rect(x0=x0, y0=y0, x1=x1, y1=y1,
#          background='blue!20',label=r'\texttt{ }', linewidth=0)

x0,y0 = C[1][0].bottomleft()
x1,y1 = C[1][0].topright()
p += Rect(x0=x0, y0=y0, x1=x1, y1=y1,
          background='blue!20',label=r'\texttt{ }', linewidth=0)

x0,y0 = C[3][1].bottomleft()
x1,y1 = C[3][1].topright()
p += Rect(x0=x0, y0=y0, x1=x1, y1=y1,
          background='blue!20',label='', linewidth=0)

x0,y0 = C[4][5].bottomleft()
x1,y1 = C[4][5].topright()
p += Rect(x0=x0, y0=y0, x1=x1, y1=y1,
          background='blue!20', label=r'\texttt{ }', linewidth=0)

x0,y0 = C[4][15].bottomleft()
x1,y1 = C[4][15].topright()
p += Rect(x0=x0, y0=y0, x1=x1, y1=y1,
          background='blue!20',label=r'\texttt{ }', linewidth=0)

# Draw table one more time to get border of C[1][4] correct
table2(p, m, rowlabel=None, collabel=None,
rownames=[r'\texttt{%s}' % r for r in range(0, 6)],
colnames=[r'\texttt{%s}' % r for r in range(0, 21)])
print(p)
\end{python}
The first output character \verb!'I'! has already been filled for you.

Characters graded are in the shaded cells.
Now write down the character (remember your single quotes!)
printed at the row number and column number.
The row and column numbering starts with 0.

%------------------------------------------------------------------------------
\newpage
\nextq
Character at row 0, column 2:
\\
\ANSWER
\begin{answercode}
w
\end{answercode}

%------------------------------------------------------------------------------
\nextq
Character at row 1, column 0:
\\
\ANSWER
\begin{answercode}
d
\end{answercode}

%------------------------------------------------------------------------------
\nextq
Character at row 3, column 1:
\\
\ANSWER
\begin{answercode}
"
\end{answercode}

%------------------------------------------------------------------------------
\nextq
Character at row 4, column 5:
\\
\ANSWER
\begin{answercode}

\end{answercode}

%------------------------------------------------------------------------------
\nextq
Character at row 4, column 15:
\\
\ANSWER
\begin{answercode}
n
\end{answercode}

%------------------------------------------------------------------------------
%-*-latex-*-
\newpage
\input{instructions}
\end{document}

