%-*-latex-*-
%-*-latex-*-
\newcommand\COURSE{ciss240}
\newcommand\ASSESSMENT{q0603}
\newcommand\ASSESSMENTTYPE{Quiz}
\newcommand\POINTS{\textwhite{xxx/xxx}}

\input{myquizpreamble}
\input{yliow}
\input{\COURSE}
%-*-latex-*-

%-*-latex-*-

%-*-latex-*-
\renewcommand\TITLE{\ASSESSMENTTYPE \ \ASSESSMENT}

\renewcommand\EMAIL{}
\renewcommand\AUTHOR{bglandis1@cougars.ccis.edu}

\textwidth=6in
\begin{document}
\topmatterthree

This is a closed-book, no compiler, 5 minute quiz.

%------------------------------------------------------------------------------
\nextq
What is the output of the following code fragment?
The repeating chunk of code appears 10 times; only 3 are shown below.
\begin{Verbatim}[frame=single,fontsize=\footnotesize]
int s = 0;
int i = 0;

if (i % 2 == 1)
{
    s = s + i;
}
i = i + 1

if (i % 2 == 1)
{
    s = s + i;
}
i = i + 1

if (i % 2 == 1)
{
    s = s + i;
}
i = i + 1

... etc ...

std::cout << s << '\n';
\end{Verbatim}
\ANSWER
\begin{answercode}

\end{answercode}

%------------------------------------------------------------------------------
\nextq
Complete the initialization of \verb!i! so that the following program
prints a random integer in the range of 50 to 60 (inclusive):
\begin{console}[commandchars=\~\!\@,fontsize=\footnotesize]
#include <iostream>
#include <cstdlib>
#include <ctime>

int main()
{
    srand((unsigned int) time(NULL));
    int i = _________________________;
    std::cout << i << '\n';
    return 0;
}
\end{console}
\ANSWER
\begin{answercode}
int i =               ;
\end{answercode}

%------------------------------------------------------------------------------
\nextq
Complete the following program so that it prints a random double in the
range of -1.0 to 3.0 (inclusive):
\begin{console}[commandchars=\~\!\@,fontsize=\footnotesize]
#include <iostream>
#include <cstdlib>
#include <ctime>

int main()
{
    srand((unsigned int) time(NULL));
    double x = _________________________;
    std::cout << x << '\n';
    return 0;
}
\end{console}
\ANSWER
\begin{answercode}
double x =               ;
\end{answercode}

%------------------------------------------------------------------------------
\nextq
Complete the following program so that it prints a random integer from
the following: 1, 2, 3, 4, 5, 20, 21, 22, 23, 24.
\begin{console}[commandchars=\~\!\@,fontsize=\footnotesize]
#include <iostream>
#include <cstdlib>
#include <ctime>

int main()
{
    srand((unsigned int) time(NULL));
    int i = _________________________;
    // write an if statement
    std::cout << i << '\n';
    return 0;
}
\end{console}
\ANSWER
\begin{answercode}
int i =       ;
if ()
{
}
\end{answercode}
(HINT: Generate a random integer 1, 2, 3, 4, 5, 6, 7, 8, 9, 10 and put it in
\verb!i!. If \verb!i! is 6, 7, 8, 9, 10, do something to \verb!i!.)

%------------------------------------------------------------------------------
%-*-latex-*-
\newpage
\input{instructions}
\end{document}

